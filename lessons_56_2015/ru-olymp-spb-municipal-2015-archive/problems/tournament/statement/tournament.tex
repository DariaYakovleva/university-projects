\begin{problem}{Турнир ФХЛ}{стандартный ввод}{стандартный вывод}{2 секунды}{256 мегабайт}

Финальный турнир Флатландской Хоккейной Лиги (ФХЛ) играется между двумя командами-лидерами сезона.
Команды играют матчи между собой до тех пор, пока одна из команд не выиграет ровно $n$ матчей.
Эта команда становится чемпионом ФХЛ. 
Каждый матч в финальном турнире заканчивается победой одной из команд, ничьих не бывает.
Видеозаписи матчей публикуются на официальном сайте ФХЛ, так что все фанаты, которые пропустили матчи,
могут посмотреть их в записи.

В этом году в финал вышли команды <<Капибары>> и <<Бурундучки>>.
Петя и Вася очень любят хоккей, но во время турнира они были на сборах по информатике. 
Теперь они решили просмотреть все матчи финального турнира в записи, скачав их с официального сайта.
Зайдя на сайт, они обнаружили, что в этом году финальный турнир ФХЛ состоял из $k$ матчей.
Скачав все видеозаписи, ребята начали их смотреть, но неожиданно поняли, что могут предсказать
итог турнира, не досмотрев все матчи. Более того, они заметили, что про некоторые матчи
они понимают, кто их выиграет, даже не начав смотреть запись.

Например, пусть $n = 3$ и $k = 4$. Петя и Вася сразу могут сделать вывод, что турнир
закончится со счетом по матчам $3:1$ или $1:3$, ведь всего будет сыграно 4 игры. Пусть первый матч закончился
победой команды <<Капибары>>, счет стал $1:0$, второй матч также закончился победой команды <<Капибары>>, счет
стал $2:0$. Теперь ребята точно знают, что победителем турнира станет команда <<Капибары>>, ведь
если бы турнир выиграла команда <<Бурундучки>>, то финальный счет был бы $2:3$ и всего
было бы сыграно 5 игр. Более того, команда <<Бурундучки>> гарантированно выиграет третий матч,
иначе окончательный счет был бы $3:0$, а команда <<Капибары>> "--- четвертый матч.

По заданным $n$, $k$ и результатам игр определите, после какой игры Петя и Вася
поймут, какая команда станет победителем турнира, а также про каждый матч
определите, знают ли ребята победителя этого матча до того, как посмотрят его.

\InputFile
Первая строка ввода содержит два целых числа: $n$ и $k$ ($1 \le n \le 100$, $n \le k \le 2n-1$).
Вторая строка ввода содержит $k$ целых чисел: $i$-е из них равно 1, если $i$-й матч
выиграла команда <<Капибары>>, либо 2, если $i$-й матч выиграла команда <<Бурундучки>>.

Гарантируется, что по итогам турнира одна из команд выиграла ровно $n$ матчей, причем
ни одна из команд не выигрывает $n$ матчей до того, как будут
сыграны все $k$ матчей, описанных во входных данных.

\OutputFile
Выведите две строки. Первая строка должна содержать одно число $z$ ($1 \le z \le k$) "--- номер
матча, после которого ребята могут однозначно определить победителя турнира.

Вторая строка должна содержать $k$ чисел, каждое из которых равно 0 или 1. Выведите 0 для
матчей, победитель которых не известен до его просмотра, и 1 для тех матчей, победителя которых
ребята могут однозначно предсказать, посмотрев все предыдущие матчи и зная числа $n$ и $k$.

\Example

\begin{example}%
\exmp{
3 4
1 1 2 1
}{
2
0 0 1 1
}%
\end{example}


\end{problem}

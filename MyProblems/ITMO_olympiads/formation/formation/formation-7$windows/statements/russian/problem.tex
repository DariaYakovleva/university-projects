\begin{problem}{Шеренга}{formation.in}{formation.out}{2 секунды}{64 мегабайта}

Галантерейщик Бонасье очень любит конфеты.
Однажды за ужином его жена Констанция выложила в ряд $n$ конфет.
У каждой конфеты есть тип $p_i$.
Констанция предложила мужу выбрать последовательность подряд идущих конфет с одним условием~--- в этой последовательности должно быть ровно два различных типа конфет.
Бонасье просит вас узнать, какое максимальное количество конфет он может взять, учитывая условие Констанции.


\InputFile
В первой строке находится одно натуральное число $n$~--- количество конфет.
Во второй строке содержатся $n$ целых чисел $p_i (1 \le p_i \le 10^9)$, где $p_i$~--- тип $i$-й конфеты.

\OutputFile
Выведите максимальное количество конфет, которые может взять Бонасье.

\begingroup
\begin{center}
\begin{flushleft}\Scoring\end{flushleft}
\renewcommand{\arraystretch}{1.5}
\begin{tabular}{|c|c|c|p{8.2cm}|}
\hline
& & \multicolumn{1}{c|}{Ограничения} & \\
\cline{3-3}
\raisebox{2.25ex}[0cm][0cm]{\parbox{1.6cm}{\begin{center}\hyphenpenalty=10000 Номер подзадачи\end{center}}} & 
\raisebox{2.25ex}[0cm][0cm]{Баллы} &
$n$ & \raisebox{2.25ex}[0cm][0cm]{\parbox{8.2cm}{\begin{center}\hyphenpenalty=10000 Комментарии\end{center}}} 
\\ \hline
1   & 30 & $1 \le n \le 100$ & Баллы начисляются, если все тесты пройдены.
\\ \hline
2   & 30 & $1 \le n \le 1000$  & Баллы начисляются, если все тесты этой и предыдущих подзадач пройдены.
\\ \hline
3   & 40 & $1 \le n \le 10^6$ & Баллы начисляются, если все тесты этой и предыдущих подзадач пройдены.
\\ \hline
\end{tabular}
\end{center}
\endgroup

Первая группа тестов состоит из тестов, для которых выполняются ограничения $n \le 100$. Баллы за эту группу начисляются только при прохождении всех тестов группы. Стоимость группы составляет 30 баллов.

Вторая группа тестов состоит из тестов, для которых выполняется ограничение $n \le 1000$.Баллы за эту группу начисляются только при прохождении всех тестов группы. Стоимость группы составляет 30 баллов.

Третья группа тестов состоит из тестов, для которых выполняется ограничение $n \le 10^6$.Баллы за эту группу начисляются только при прохождении всех тестов группы. Стоимость группы составляет 40 баллов.

\Example

\begin{example}
\exmp{6
3 3 1 2 2 1
}{4
}%
\end{example}

\Note
В первом примере на столе лежит три типа конфет 1, 2 и 3.
Бонасье может взять первые три конфеты с типами 3, 3, 1, а может взять последние четыре конфеты 1, 2, 2, 1. Значит, максимальное количество конфет, которое он может взять равно четырем.

\end{problem}

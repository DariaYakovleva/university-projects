
\thispagestyle{empty}
\begin{center}
{\Large\bfseries Министерство образования и науки Российской Федерации}
Федеральное государственное автономное образовательное учреждение
высшего образования
{\bfseries «Санкт-Петербургский национальный исследовательский университет
информационных технологий, механики и оптики»}

\vspace{1cm}

УТВЕРЖДАЮ
Проректор по УМР

Шехонин А.А.
``''20г.
м.п.

{\hrule height 2pt}

\newlength{\pretext}

\newcommand{\titleunderline}[2]{%
\setlength{\remaining}{\textwidth-\pretext}%
\noindent\textsc{#1}\underline{\makebox[\remaining]{\textsc{#2}}}\par}

\titleunderline{\textbf{Направление подготовки}}{010400 «Прикладная математика и информатика»}\vspace{12pt}
\titleunderlineleft{\textbf{Профиль подготовки бакалавра}}{010400.62.01 «Математические модели и алгоритмы}
\titleunderlinecont{в разработке программного обеспечения»}

\begin{tabular}{!{\VRule}c!{\VRule}c!{\VRule}c!{\VRule}c!{\VRule}c!{\VRule}c!{\VRule}c!{\VRule}}\HLine
\bfseries Семестр &
\bfseries \pb{Трудо-емкость, час.} &
\bfseries \pb{Лек-ций, час.} &
\bfseries \pb{Практич. занятий, час.} &
\bfseries \pb{Лаборат. работ, час.} &
\bfseries \pb{СРС, час.} &
\bfseries \pb{Форма промежуточного контроля (экзамен/зачет)}\HLine
7 & 102 & 34 & 0 & 34 & 34 & Экзамен\HLine
8 & 82 & 14 & 0 & 28 & 40 & Экзамен\HLine

Итого: & 184 & 48 & 0 & 62 & 74 & \HLine
\end{tabular}

\vfill

Санкт-Петербург,
2014

\newpage
\section*{РАБОЧАЯ ПРОГРАММА ДИСЦИПЛИНЫ}

\paragraph{Разделы рабочей программы}
\begin{enumerate}
\item Цели освоения дисциплины
\item Место дисциплины в структуре ООП ВПО
\item Структура и содержание дисциплины
\item Формы контроля освоения дисциплины
\item Учебно-методическое и информационное обеспечение дисциплины
\item Материально-техническое обеспечение дисциплины
\end{enumerate}

\paragraph{Приложения к рабочей программе дисциплины}
{
\setenumerate[1]{label={Приложение \arabic{enumi}.}, fullwidth, itemindent=\parindent, listparindent=\parindent} 
\begin{enumerate}
\item Аннотация рабочей программы
\item Технологии и формы преподавания 
\item Технологии и формы обучения 
\item Оценочные средства и методики их применения
\item Таблица планирования результатов обучения
\end{enumerate}
}

Программа составлена в соответствии с требованиями ОС НИУ ИТМО по направлению подготовки: 010400 «Прикладная математика и информатика».

{\parindent0pt
Программу составили:
кафедра КТ  Корнеев Г. А., кандидат технических наук, доцент  Маврин П. Ю., старший преподаватель
Зав. кафедрой:
 Парфенов В. Г., доктор технических наук, профессор
Эксперт(ы):  Шопырин Д. Г., генеральный директор ООО "ВСВН Лаб"
Программа одобрена на заседании УМК факультета Информационных Технологий и Программирования.
Председатель УМК ИТиП:
 Харченко Т. В.
}

{\parindent0pt
\textbf{знания}:
\begin{itemize}
\item на уровне представлений:
\begin{itemize}
\item представление об основах принципах построения сетей на основе многоуровневых архитектур и межсетевого взаимодействия,\item представление об основных локальных и глобальных протоколах канального уровня,\item представление о современных стандартах беспроводной передачи данных.,\item представление о консорциуме Интернет 2, характеристиках разрабатываемой им сети.,\item представление о современных стандартах беспроводной передачи данных.,
\end{itemize}
\item на уровне воспроизведения:
\begin{itemize}
\item воспроизведение общих принципов работы сетевого, транспортного и прикладного уровня стека сетевых протоколов,
\end{itemize}
\item на уровне понимания:
\begin{itemize}
\item понимание основных возможностей, принципов построения и распространенных архитектур современных информационных сетей,,\item знание о плезиохронной и синхронной цифровых иерархиях, их историей и назначениями,
\end{itemize}
\end{itemize}
\textbf{умения}:
\begin{itemize}
\item теоретические:
\begin{itemize}
\item анализ, оценка и выбор из существующих технических систем, выбору того или иного технического решения в зависимости от поставленной цели,\item выбор архитектуры компьютерной сети в зависимости от поставаленной цели,
\end{itemize}
\item практические:
\begin{itemize}
\item по проектированию, развертыванию и управлению современными информационными сетями на основе типичного аппаратного и программного обеспечения,
\end{itemize}
\end{itemize}
\textbf{навыки}:
\begin{itemize}
\item навык командной работы, как на этапе проектирования, так и на этапе реализации компонентов информационных сетей..
\end{itemize}

Перечисленные РО являются основой для формирования следующих компетенций:
\begin{itemize}
\item профессиональных: \begin{itemize}
\item ПК.3 — способностью понимать и применять в исследовательской и прикладной деятельности современный математический аппарат,\item ПК.4 — способностью в составе научно-исследовательского и производственного коллектива решать задачи профессиональной деятельности,\item ПК.5 — способностью критически переосмысливать накопленный опыт, изменять при необходимости вид и характер своей профессиональной деятельности,\item ПК.7 — способностью собирать, обрабатывать и интерпретировать данные современных научных исследований, необходимые для формирования выводов по соответствующим научным, профессиональным, социальным и этическим проблемам,\item ПК.12 — способностью составлять и контролировать план выполняемой работы, планировать необходимые для выполнения работы ресурсы, оценивать результаты собственной работы,
\end{itemize}
\end{itemize}
}

\newpage
\section{МЕСТО ДИСЦИПЛИНЫ В СТРУКТУРЕ ООП ВПО}
Дисциплина «Компьютерные сети» является частью вариативной части профессионального цикла дисциплин.
Необходимыми условиями для освоения дисциплины являются:
\begin{itemize}
\item знания:
\begin{itemize}
\item знание базовых приемов, используемых при проектировании алгоритмов и структур данных,\item воспроизведение базовых алгоритмов, основанных на использовании основных структур данных,\item понимание базовых структур данных и операций над ними,\item технический английский,
\end{itemize}
\item умения:
\begin{itemize}
\item особенности сетевого взаимодействия операционных систем,
\end{itemize}
\item навыки:
\begin{itemize}
\item алгоритмическое мышление.
\end{itemize}
\end{itemize}


Содержание дисциплины является логическим продолжением содержания дисциплин: \begin{itemize}
\item Б.2.1.1 «Математический анализ»\item Б.2.1.2 «Алгебра и геометрия»\item Б.2.1.3 «Физика»\item Б.2.2.1.4 «Математическая физика»\item Б.2.2.1.5 «Функциональный анализ»\item Б.2.2.1.6 «Концепции современного естествознания»\item Б.2.2.1.7 «Численные методы»\item Б.3.1.1 «Безопасность жизнедеятельности»\item Б.3.1.10 «Языки программирования»\item Б.3.1.11 «Операционные системы»\item Б.3.1.2 «Дискретная математика»\item Б.3.1.3 «Алгоритмы и структуры данных»\item Б.3.1.4 «Теория формальных языков»\item Б.3.1.5 «Методы трансляции»\item Б.3.1.6 «Теория вероятностей и математическая статистика»\item Б.3.1.8 «Введение в программирование и ЭВМ»\item Б.3.1.9 «Технологии программирования»\item Б.3.2.1.1 «Автоматное программирование»\item Б.3.2.1.2 «Вычислительная геометрия»\item Б.3.2.1.3 «Параллельное программирование»\item Б.3.2.1.4 «Теория вычислительной сложности»\item Б.3.2.2.1 «Парадигмы программирования»\item Б.3.2.2.1 «Язык программирования Java»\item Б.3.2.2.2 «Алгоритмы в математике»\item Б.3.2.2.2 «Специальный семинар»\item Б.3.2.2.5 «Практикум на ЭВМ»\item Б.3.2.2.5 «Специальный семинар»\item Б.5.1 «Производственная практика»
\end{itemize} и служит основой для освоения дисциплин: \begin{itemize}
\item Б.2.2.1.8 «Теория игр и исследования операций»\item Б.3.1.13 «Методы оптимизации»\item Б.5.2 «Преддипломная практика»
\end{itemize}

\newpage
В таблице приведены предшествующие и последующие дисциплины, направленные на формирование компетенций, заявленных в разделе «Цели освоения дисциплины»:

\begin{longtable}{|c|p{0.15\textwidth}|p{0.35\textwidth}|p{0.35\textwidth}|}\hline
№ п/п &
\multicolumn{1}{c|}{\pb{Наименованиекомпетенции}} &
\multicolumn{1}{c|}{Предшествующие дисциплины} &
\multicolumn{1}{c|}{\pb{Последующие дисциплины(группы дисциплин)}}\hline
\multicolumn{4}{|l|}{\textit{Профессиональные компетенции}}\hline 1 & ПК.3 & Б.2.1.1 «Математический анализ», Б.2.1.2 «Алгебра и геометрия», Б.2.1.3 «Физика», Б.2.2.1.4 «Математическая физика», Б.2.2.1.5 «Функциональный анализ», Б.2.2.1.6 «Концепции современного естествознания», Б.2.2.1.7 «Численные методы», Б.3.1.1 «Безопасность жизнедеятельности», Б.3.1.10 «Языки программирования», Б.3.1.11 «Операционные системы», Б.3.1.2 «Дискретная математика», Б.3.1.3 «Алгоритмы и структуры данных», Б.3.1.4 «Теория формальных языков», Б.3.1.5 «Методы трансляции», Б.3.1.6 «Теория вероятностей и математическая статистика», Б.3.1.8 «Введение в программирование и ЭВМ», Б.3.1.9 «Технологии программирования», Б.3.2.1.1 «Автоматное программирование», Б.3.2.1.2 «Вычислительная геометрия», Б.3.2.1.3 «Параллельное программирование», Б.3.2.1.4 «Теория вычислительной сложности», Б.3.2.2.1 «Парадигмы программирования», Б.3.2.2.1 «Язык программирования Java», Б.3.2.2.2 «Алгоритмы в математике», Б.3.2.2.2 «Специальный семинар», Б.3.2.2.5 «Практикум на ЭВМ», Б.3.2.2.5 «Специальный семинар», Б.5.1 «Производственная практика» & Б.2.2.1.8 «Теория игр и исследования операций», Б.3.1.13 «Методы оптимизации», Б.5.2 «Преддипломная практика»\hline
2 & ПК.4 & Б.2.1.1 «Математический анализ», Б.2.1.2 «Алгебра и геометрия», Б.2.1.3 «Физика», Б.2.2.1.4 «Математическая физика», Б.2.2.1.5 «Функциональный анализ», Б.2.2.1.6 «Концепции современного естествознания», Б.2.2.1.7 «Численные методы», Б.3.1.1 «Безопасность жизнедеятельности», Б.3.1.10 «Языки программирования», Б.3.1.11 «Операционные системы», Б.3.1.2 «Дискретная математика», Б.3.1.3 «Алгоритмы и структуры данных», Б.3.1.4 «Теория формальных языков», Б.3.1.5 «Методы трансляции», Б.3.1.6 «Теория вероятностей и математическая статистика», Б.3.1.8 «Введение в программирование и ЭВМ», Б.3.1.9 «Технологии программирования», Б.3.2.1.1 «Автоматное программирование», Б.3.2.1.2 «Вычислительная геометрия», Б.3.2.1.3 «Параллельное программирование», Б.3.2.1.4 «Теория вычислительной сложности», Б.3.2.2.1 «Парадигмы программирования», Б.3.2.2.1 «Язык программирования Java», Б.3.2.2.2 «Алгоритмы в математике», Б.3.2.2.2 «Специальный семинар», Б.3.2.2.5 «Практикум на ЭВМ», Б.3.2.2.5 «Специальный семинар», Б.5.1 «Производственная практика» & Б.2.2.1.8 «Теория игр и исследования операций», Б.3.1.13 «Методы оптимизации», Б.5.2 «Преддипломная практика»\hline
3 & ПК.5 & Б.2.1.1 «Математический анализ», Б.2.1.2 «Алгебра и геометрия», Б.2.1.3 «Физика», Б.2.2.1.4 «Математическая физика», Б.2.2.1.5 «Функциональный анализ», Б.2.2.1.6 «Концепции современного естествознания», Б.2.2.1.7 «Численные методы», Б.3.1.1 «Безопасность жизнедеятельности», Б.3.1.10 «Языки программирования», Б.3.1.11 «Операционные системы», Б.3.1.2 «Дискретная математика», Б.3.1.3 «Алгоритмы и структуры данных», Б.3.1.4 «Теория формальных языков», Б.3.1.5 «Методы трансляции», Б.3.1.6 «Теория вероятностей и математическая статистика», Б.3.1.8 «Введение в программирование и ЭВМ», Б.3.1.9 «Технологии программирования», Б.3.2.1.1 «Автоматное программирование», Б.3.2.1.2 «Вычислительная геометрия», Б.3.2.1.3 «Параллельное программирование», Б.3.2.1.4 «Теория вычислительной сложности», Б.3.2.2.1 «Парадигмы программирования», Б.3.2.2.1 «Язык программирования Java», Б.3.2.2.2 «Алгоритмы в математике», Б.3.2.2.2 «Специальный семинар», Б.3.2.2.5 «Практикум на ЭВМ», Б.3.2.2.5 «Специальный семинар», Б.5.1 «Производственная практика» & Б.2.2.1.8 «Теория игр и исследования операций», Б.3.1.13 «Методы оптимизации», Б.5.2 «Преддипломная практика»\hline
4 & ПК.7 & Б.2.1.1 «Математический анализ», Б.2.1.2 «Алгебра и геометрия», Б.2.1.3 «Физика», Б.2.2.1.4 «Математическая физика», Б.2.2.1.5 «Функциональный анализ», Б.2.2.1.6 «Концепции современного естествознания», Б.2.2.1.7 «Численные методы», Б.3.1.1 «Безопасность жизнедеятельности», Б.3.1.10 «Языки программирования», Б.3.1.11 «Операционные системы», Б.3.1.2 «Дискретная математика», Б.3.1.3 «Алгоритмы и структуры данных», Б.3.1.4 «Теория формальных языков», Б.3.1.5 «Методы трансляции», Б.3.1.6 «Теория вероятностей и математическая статистика», Б.3.1.8 «Введение в программирование и ЭВМ», Б.3.1.9 «Технологии программирования», Б.3.2.1.1 «Автоматное программирование», Б.3.2.1.2 «Вычислительная геометрия», Б.3.2.1.3 «Параллельное программирование», Б.3.2.1.4 «Теория вычислительной сложности», Б.3.2.2.1 «Парадигмы программирования», Б.3.2.2.1 «Язык программирования Java», Б.3.2.2.2 «Алгоритмы в математике», Б.3.2.2.2 «Специальный семинар», Б.3.2.2.5 «Практикум на ЭВМ», Б.3.2.2.5 «Специальный семинар», Б.5.1 «Производственная практика» & Б.2.2.1.8 «Теория игр и исследования операций», Б.3.1.13 «Методы оптимизации», Б.5.2 «Преддипломная практика»\hline
5 & ПК.12 & Б.2.1.1 «Математический анализ», Б.2.1.2 «Алгебра и геометрия», Б.2.1.3 «Физика», Б.2.2.1.4 «Математическая физика», Б.2.2.1.5 «Функциональный анализ», Б.2.2.1.6 «Концепции современного естествознания», Б.2.2.1.7 «Численные методы», Б.3.1.1 «Безопасность жизнедеятельности», Б.3.1.10 «Языки программирования», Б.3.1.11 «Операционные системы», Б.3.1.2 «Дискретная математика», Б.3.1.3 «Алгоритмы и структуры данных», Б.3.1.4 «Теория формальных языков», Б.3.1.5 «Методы трансляции», Б.3.1.6 «Теория вероятностей и математическая статистика», Б.3.1.8 «Введение в программирование и ЭВМ», Б.3.1.9 «Технологии программирования», Б.3.2.1.1 «Автоматное программирование», Б.3.2.1.2 «Вычислительная геометрия», Б.3.2.1.3 «Параллельное программирование», Б.3.2.1.4 «Теория вычислительной сложности», Б.3.2.2.1 «Парадигмы программирования», Б.3.2.2.1 «Язык программирования Java», Б.3.2.2.2 «Алгоритмы в математике», Б.3.2.2.2 «Специальный семинар», Б.3.2.2.5 «Практикум на ЭВМ», Б.3.2.2.5 «Специальный семинар», Б.5.1 «Производственная практика» & Б.2.2.1.8 «Теория игр и исследования операций», Б.3.1.13 «Методы оптимизации», Б.5.2 «Преддипломная практика»\hline

\end{longtable}

\newpage
\section{СТРУКТУРА И СОДЕРЖАНИЕ ДИСЦИПЛИНЫ}
Общая трудоемкость дисциплины составляет 5 зачетных единиц, 184 часов.

\begin{center}
\begin{longtable}{|c|c|p{0.25\textwidth}|p{1.4cm}|p{1.4cm}|p{1.4cm}|p{1.4cm}|p{1.4cm}|}\hline
\multicolumn{1}{|c|}{\multirow{2}{*}{\pb{\bfseries№модуля}}} &
\multicolumn{1}{c|}{\multirow{2}{*}{\pb{\bfseries№раздела}}} &
\multicolumn{1}{c|}{\multirow{2}{*}{\pb{\bfseriesНаименованиеразделадисциплины}}} &
\multicolumn{5}{c|}{\pb{\bfseries{}Виды учебной нагрузки и их трудоемкость, часы}}\cline{4-8}
& & &
\multicolumn{1}{c|}{\bfseries\begin{sideways}Лекции\end{sideways}} &
\multicolumn{1}{c|}{\bfseries\begin{sideways}\pb{Практическиезанятия}\end{sideways}} &
\multicolumn{1}{c|}{\bfseries\begin{sideways}\pb{Лабораторныеработы}\end{sideways}} &
\multicolumn{1}{c|}{\bfseries\begin{sideways}СРС\end{sideways}} &
\multicolumn{1}{c|}{\bfseries\begin{sideways}Всего часов\end{sideways}}\hline
13 & 1 & Общие принципы построения сетей, физические и канальные протоколы & 18 & 0 & 18 & 18 & 54\hline
14 & 2 & Протоколы сетевого, транспортного и прикладного уровней & 16 & 0 & 16 & 16 & 48\hline
15 & 3 & Иерархии & 7 & 0 & 14 & 21 & 42\hline
16 & 4 & Frame relay, ATM, VLAN, PPPoE, QoS, IPv6. & 2 & 0 & 9 & 7 & 18\hline
16 & 5 & Стандарты беспроводной передачи данных & 5 & 0 & 5 & 12 & 22\hline

\end{longtable}
\end{center}

\subsection{Содержание (дидактика) дисциплины}
% XXX Тут было решено положить болт на дидактические еденицы. Спасибо за понимание.

\begin{description}
\item[Раздел 1.] «Общие принципы построения сетей, физические и канальные протоколы».\item[Раздел 2.] «Протоколы сетевого, транспортного и прикладного уровней».\item[Раздел 3.] «Иерархии».\item[Раздел 4.] «Frame relay, ATM, VLAN, PPPoE, QoS, IPv6.».\item[Раздел 5.] «Стандарты беспроводной передачи данных».
\end{description}

\subsection{Лекции}

\begin{center}
\begin{longtable}{|c|c|c|p{0.6\textwidth}|}\hline
\multicolumn{1}{|c|}{\pb{\bfseries№п/п}} &
\multicolumn{1}{c|}{\pb{\bfseries №раздела}} &
\multicolumn{1}{c|}{\pb{\bfseries Объем,часов}} &
\multicolumn{1}{c|}{\bfseries Тема лекции} \hline
1 & 1 & 4 & Среды передачи данных\hline
2 & 1 & 4 & Уровни модели OSI\hline
3 & 1 & 5 & Физические протоколы передачи данных\hline
4 & 1 & 5 & Канальные протоколы локальных сетей\hline
5 & 2 & 4 & Архитектура стека TCP/IP. Адресная информация\hline
6 & 2 & 4 & Сетевой и транспортный уровень стека TCP/IP\hline
7 & 2 & 4 & Межсетевое взаимодействие\hline
8 & 2 & 4 & Прикладные протоколы стека TCP/IP\hline
9 & 3 & 3 & PDH\hline
10 & 3 & 4 & SDH\hline
11 & 4 & 1 & Frame relay, ATM, VLAN\hline
12 & 4 & 1 & PPPoE, QoS, IPv6\hline
13 & 5 & 1 & WiFi -- IEEE 802.11\hline
14 & 5 & 1 & WiMAX -- IEEE 802.16e\hline
15 & 5 & 1 & Bluetooth -- IEEE 802.15.1\hline
16 & 5 & 1 & NFC - ISO 13157\hline
17 & 5 & 1 & LTE\hline

\multicolumn{2}{|c|}{Итого:} & 48 & \hline
\end{longtable}
\end{center}


\subsection{Практические занятия}
Не предусмотрены.

\subsection{Лабораторные работы}

\begin{center}
\begin{longtable}{|c|c|c|p{0.4\textwidth}|p{0.2\textwidth}|}\hline
\multicolumn{1}{|c|}{\pb{\bfseries№п/п}} &
\multicolumn{1}{c|}{\pb{\bfseries №раздела}} &
\multicolumn{1}{c|}{\pb{\bfseries Трудоемкость,часов}} &
\multicolumn{1}{c|}{\pb{\bfseries Наименование лабораторнойработы}} &
\multicolumn{1}{c|}{\pb{\bfseries Наименованиелаборатории}} \hline
1 & 1 & 9 & Эмуляция топологий & Компьютерный класс\hline
2 & 1 & 9 & Физическое кодирование сигнала & Компьютерный класс\hline
3 & 2 & 16 & Настройка сети & Компьютерный класс\hline
4 & 3 & 14 & Иерархии & Компьютерный класс\hline
5 & 4 & 4 & IPv6 & Компьютерный класс\hline
6 & 4 & 5 & Протоколы маршрутизации & Компьютерный класс\hline
7 & 5 & 5 & CDMA & Компьютерный класс\hline

\multicolumn{2}{|c|}{Итого:} & 62 & & \hline
\end{longtable}
\end{center}


\subsection{Самостоятельная работа студента}

\begin{center}
\begin{longtable}{|c|c|c|p{0.6\textwidth}|}\hline
\multicolumn{1}{|c|}{\pb{\bfseries№п/п}} &
\multicolumn{1}{c|}{\pb{\bfseries №раздела}} &
\multicolumn{1}{c|}{\pb{\bfseries Трудоемкость,часов}} &
\multicolumn{1}{c|}{\pb{\bfseries Вид СРС}} \hline
1 & 1 & 12 & Подготовка к лекциям\hline
2 & 1 & 6 & Выполнение лабораторных работ\hline
3 & 2 & 12 & Подготовка к лекциям\hline
4 & 2 & 4 & Выполнение лабораторных работ\hline
5 & 3 & 14 & Подготовка к лекциям\hline
6 & 3 & 7 & Выполнение лабораторных работ\hline
7 & 4 & 3 & Подготовка к лекциям\hline
8 & 4 & 4 & Выполнение лабораторных работ\hline
9 & 5 & 10 & Подготовка к лекциям\hline
10 & 5 & 2 & Выполнение лабораторных работ\hline

\multicolumn{2}{|c|}{Итого:} & 74 & \hline
\end{longtable}
\end{center}


\subsection{Домашние задания, типовые расчеты и т.п.}
Не предусмотрены.

\subsection{Рефераты}
Не предусмотрены.

\subsection{Курсовые работы по дисциплине}
Не предусмотрены.

\newpage
\section{ФОРМЫ КОНТРОЛЯ ОСВОЕНИЯ ДИСЦИПЛИНЫ}

Текущий контроль успеваемости по дисциплине и промежуточная аттестация студентов по результатам семестра осуществляются в соответствии с положением о проведении текущего контроля успеваемости и промежуточной аттестации студентов НИУ ИТМО.

\textbf{Текущая аттестация} студентов производится лектором и преподавателем (преподавателями), ведущими лабораторные работы и практические занятия по дисциплине в следующих формах:
\begin{itemize}


\item выполнение лабораторных работ;
\item защита лабораторных работ;
\item отдельно оцениваются личностные качества студента.
\end{itemize}

\textbf{Рубежная аттестация} студентов производится по окончании модуля в следующих формах:
\begin{itemize}
\item защита лабораторных работ.
\end{itemize}

\textbf{Промежуточный контроль} по результатам семестра по дисциплине проходит:
\begin{itemize}
\item в форме устного экзамена в семестре №7;\item в форме устного экзамена в семестре №8.
\end{itemize}

Фонды оценочных средств, включающие типовые задания, контрольные работы, тесты и методы контроля, позволяющие оценить РО по данной дисциплине, включены в состав УМК дисциплины и перечислены в Приложении 4.
Критерии оценивания, перечень контрольных точек и таблица планирования результатов обучения приведены в Приложениях 4 и 5 к Рабочей программе.

\newpage
\section{УЧЕБНО-МЕТОДИЧЕСКОЕ И ИНФОРМАЦИОННОЕ ОБЕСПЕЧЕНИЕ ДИСЦИПЛИНЫ}

\setenumerate[1]{label={\alph{enumi})}} 
\setenumerate[2]{label={[\arabic{enumii}]},ref={\arabic{enumii}}} 
\begin{enumerate}
\item основная литература:
\begin{enumerate}
\item \label{olifer} Олифер В., Олифер Н. Компьютерные сети. Принципы, технологии, протоколы. — Питер, 2014. \item \label{tanenbaum-eng} Tanenbaum A. S., Wetherall D. J. Computer Networks (5th Edition). — Prentice Hall, 2010.
\end{enumerate}
 \item дополнительная литература:

 \item программное обеспечение, Интернет-ресурсы, электронные библиотечные системы:
\begin{enumerate}[resume]
\item \label{www1} Designing and Building Parallel Programs, by Ian Foste \url{http://www-unix.mcs.anl.gov/dbpp/}
\end{enumerate}
\end{enumerate}

\newpage
\section{МАТЕРИАЛЬНО-ТЕХНИЧЕСКОЕ ОБЕСПЕЧЕНИЕ ДИСЦИПЛИНЫ}

\setenumerate[1]{label={\arabic{enumi}.}} 
\setenumerate[2]{label={\alph{enumii}.}} 

\begin{enumerate}
\item Лекционные занятия:
\begin{enumerate}
\item аудитория, оснащенная маркерной доской, \item комплект электронных презетраций/слайдов, \item презентационная техника (проектор, экран, компьютер/ноутбук).
\end{enumerate}
\item Лабораторные работы
\begin{enumerate}
\item компьютерный класс.
\end{enumerate}
\item Прочее
\begin{enumerate}
\item рабочее место преподавателя, оснащенное компьютером с доступом в Интернет, \item рабочие места студентов, оснащенные компьютерами с доступом в Интернет, предназначенные для работы в электронной образовательной среде.
\end{enumerate}
\end{enumerate}

\newpage
\begin{flushright}
\textbf{Приложение 1
к рабочей программе дисциплины
«Компьютерные сети»}
\end{flushright}
\section*{Аннотация рабочей программы}

Дисциплина «Компьютерные сети» является частью вариативной части профессионального цикла дисциплин подготовки студентов по направлению подготовки Прикладная математика и информатика.
Дисциплина реализуется на факультете ИТиП НИУ ИТМО кафедрой КТ.

Дисциплина нацелена на формирование профессиональных компетенций: ПК.3, ПК.4, ПК.5, ПК.7, ПК.12 выпускника. Дисциплина относится к циклу дисциплин общей профессиональной подготовки. Для изучения дисциплины обучающиеся должны быть знакомы с основными принципами программирования, разделами физики, описывающим распространение сигналов, основы кодирования и иметь навыки работы на компьютере и в Интернете. Знания и умения, полученные при изучении дисциплины, используются при защитах курсовых и выпускной квалификационной работ, прохождении практик.

Преподавание дисциплины предусматривает следующие формы организации учебного процесса: лекции, лабораторные работы, самостоятельная работа студента и консультации.

Программой дисциплины предусмотрены следующие виды контроля: текущий контроль успеваемости в форме
проверки выполнения лабораторных работ, защиты лабораторных работ (тестирований), рубежный контроль в форме защит лабораторных работ (тестирований) и промежуточный контроль в форме экзаменов.

Общая трудоемкость освоения дисциплины составляет 5 зачетных единиц, 184 часов. Программой дисциплины предусмотрены: лекционные   (48 часов), лабораторные (62 часов) работы и самостоятельная работа студента (74 часов).

\newpage
\begin{flushright}
\textbf{Приложение 2
к рабочей программе дисциплины
«Компьютерные сети»}
\end{flushright}

\section*{ТЕХНОЛОГИИ И ФОРМЫ ПРЕПОДАВАНИЯ
Рекомендации по организации и технологиям обучения для преподавателей}

\def\thesubsection{\Roman{subsection}}

\subsection{Образовательные технологии}

{\parindent0pt
Преподавание дисциплины ведется с применением следующих видов образовательных технологий:
\textbf{Информационные технологии}: использование электронных образовательных ресурсов при подготовке к лекциям и лабораторным работам разделов 1, 2, 3, 4 и 5.

%TODO: сделать работу в команде и прочую муть
% 3. Case-study, 4. Игра, 5. Проблемное обучение, 6. Контекстное обучение, 7. Обучение на основе опыта, 8. Индивидуальное обучение, 9. Междисциплинарное обучение, 10. Опережающая самостоятельная работа
}

\subsection{Виды и содержание учебных занятий}

\subsubsection{Раздел 1. «Общие принципы построения сетей, физические и канальные протоколы»}

{\parindent0pt
\setdescription{leftmargin=\parindent,labelindent=1cm}
\setitemize[1]{leftmargin=1.5cm}

\textbf{Теоретические занятия (лекции)— 18 часов.}
\begin{description}
\item[Лекция 1.] «Среды передачи данных». Информационная лекция. Рассматриваются следующие вопросы: \begin{itemize}
\item история сред передачи данных\item передача данных по металлическим проводам\item передача данных по многожильным проводам\item передача данных по оптоволокну
\end{itemize}\item[Лекция 2.] «Уровни модели OSI». Информационная лекция. Рассматриваются следующие вопросы: \begin{itemize}
\item физический уровень\item канальный уровень\item сетевой уровень\item транспортный уровень\item сеансовый уровень\item представительский уровень\item прикладной уровень
\end{itemize}\item[Лекция 3.] «Физические протоколы передачи данных». Информационная лекция. Рассматриваются следующие вопросы: \begin{itemize}
\item общие принципы передачи данных в сигнальной форме\item характеристики каналов связи\item физическое кодирование\item логическое кодирование
\end{itemize}\item[Лекция 4.] «Канальные протоколы локальных сетей». Информационная лекция. Рассматриваются следующие вопросы: \begin{itemize}
\item сети Ethernet, топологии, стандарты физического и канального уровней\item сети FDDI, топология, стандарт физического и канального уровней
\end{itemize}
\end{description}




\textbf{Лабораторный практикум— 18 часов, 2 работ.}
\begin{description}
\item[Лабораторная работа 1.] «Эмуляция топологий». Выполняется индивидуально в лаборатории «Компьютерный класс». Цели работы: \begin{itemize}
\item реализовать модули для сетевого взиамодействия, реализующие топологию шина на имеющейся сети\item реализовать модули для сетевого взиамодействия, реализующие топологию кольцо на имеющейся сети\item реализовать модули для сетевого взиамодействия, реализующие топологию звезда на имеющейся сети\item реализовать модули для сетевого взиамодействия, реализующие топологию двойное кольцо на имеющейся сети\item реализовать модули для сетевого взиамодействия, реализующие топологию решетка на имеющейся сети\item реализовать модули для сетевого взиамодействия, реализующие топологию дерево на имеющейся сети
\end{itemize}\item[Лабораторная работа 2.] «Физическое кодирование сигнала». Выполняется индивидуально в лаборатории «Компьютерный класс». Цели работы: \begin{itemize}
\item реализовать визуализатор NRZ кодирования\item реализовать визуализатор NRZI кодирования\item реализовать визуализатор манчестерского кодирования\item реализовать визуализатор RZ кодирования\item реализовать визуализатор MLT-3 кодирования\item реализовать визуализатор 4B3T кодирования
\end{itemize}
\end{description}

\textbf{Управление самостоятельной работой студента.}
\begin{itemize}
\item Консультации по выполнению лабораторных работ.
\end{itemize}
}


\subsubsection{Раздел 2. «Протоколы сетевого, транспортного и прикладного уровней»}

{\parindent0pt
\setdescription{leftmargin=\parindent,labelindent=1cm}
\setitemize[1]{leftmargin=1.5cm}

\textbf{Теоретические занятия (лекции)— 16 часов.}
\begin{description}
\item[Лекция 5.] «Архитектура стека TCP/IP. Адресная информация». Информационная лекция. Рассматриваются следующие вопросы: \begin{itemize}
\item Архитектура стека TCP/IP, потоки данных через стек\item Адресация на сетевом, транспортном и прикладном уровне
\end{itemize}\item[Лекция 6.] «Сетевой и транспортный уровень стека TCP/IP». Информационная лекция. Рассматриваются следующие вопросы: \begin{itemize}
\item Назначение и архитектура протокола IP v 4\item Особенности протокола IP v 6\item работа протокола UDP\item работа протокола TCP
\end{itemize}\item[Лекция 7.] «Межсетевое взаимодействие». Информационная лекция. Рассматриваются следующие вопросы: \begin{itemize}
\item Статическая маршрутизация в IP сетях\item Динамическая маршрутизация в IP сетях\item Трансляция адресов (NAT)\item Проксирование
\end{itemize}\item[Лекция 8.] «Прикладные протоколы стека TCP/IP». Информационная лекция. Рассматриваются следующие вопросы: \begin{itemize}
\item Протокол DNS\item Протоколы почтового обмена\item Протокол HTTP
\end{itemize}
\end{description}




\textbf{Лабораторный практикум— 16 часов, 1 работ.}
\begin{description}
\item[Лабораторная работа 3.] «Настройка сети». Выполняется индивидуально в лаборатории «Компьютерный класс». Цели работы: \begin{itemize}
\item Счисление адресов IP сетей\item Конфигурирование маршрутизатора
\end{itemize}
\end{description}

\textbf{Управление самостоятельной работой студента.}
\begin{itemize}
\item Консультации по выполнению лабораторных работ.
\end{itemize}
}


\subsubsection{Раздел 3. «Иерархии»}

{\parindent0pt
\setdescription{leftmargin=\parindent,labelindent=1cm}
\setitemize[1]{leftmargin=1.5cm}

\textbf{Теоретические занятия (лекции)— 7 часов.}
\begin{description}
\item[Лекция 9.] «PDH». Информационная лекция. Рассматриваются следующие вопросы: \begin{itemize}
\item предпосылки создания\item первые реализации\item устройство\item особенности протоколов
\end{itemize}\item[Лекция 10.] «SDH». Информационная лекция. Рассматриваются следующие вопросы: \begin{itemize}
\item предпосылки создания\item первые реализации\item устройство\item особенности протоколов\item сравнение PDH и SDH
\end{itemize}
\end{description}




\textbf{Лабораторный практикум— 14 часов, 1 работ.}
\begin{description}
\item[Лабораторная работа 4.] «Иерархии». Выполняется индивидуально в лаборатории «Компьютерный класс». Цели работы: \begin{itemize}
\item разработать визуализатор метода PDH\item разработать визуализатор метода SDH
\end{itemize}
\end{description}

\textbf{Управление самостоятельной работой студента.}
\begin{itemize}
\item Консультации по выполнению лабораторных работ.
\end{itemize}
}


\subsubsection{Раздел 4. «Frame relay, ATM, VLAN, PPPoE, QoS, IPv6.»}

{\parindent0pt
\setdescription{leftmargin=\parindent,labelindent=1cm}
\setitemize[1]{leftmargin=1.5cm}

\textbf{Теоретические занятия (лекции)— 2 часов.}
\begin{description}
\item[Лекция 11.] «Frame relay, ATM, VLAN». Информационная лекция. Рассматриваются следующие вопросы: \begin{itemize}
\item Frame relay\item ATM\item VLAN
\end{itemize}\item[Лекция 12.] «PPPoE, QoS, IPv6». Информационная лекция. Рассматриваются следующие вопросы: \begin{itemize}
\item PPPoE\item QoS\item IPv6
\end{itemize}
\end{description}




\textbf{Лабораторный практикум— 9 часов, 2 работ.}
\begin{description}
\item[Лабораторная работа 5.] «IPv6». Выполняется индивидуально в лаборатории «Компьютерный класс». Цели работы: \begin{itemize}
\item Студенты должны продемонстрировать знание способов перевода адресов IPv4 в IPv6 и обратно
\end{itemize}\item[Лабораторная работа 6.] «Протоколы маршрутизации». Выполняется индивидуально в лаборатории «Компьютерный класс». Цели работы: \begin{itemize}
\item реализовать визуализатор маршрутизатора, использующего протокол маршрутизации RIP\item реализовать визуализатор маршрутизатора, использующего протокол маршрутизации IGRP\item реализовать визуализатор маршрутизатора, использующего протокол маршрутизации BGP\item реализовать визуализатор маршрутизатора, использующего протокол маршрутизации EIGRP
\end{itemize}
\end{description}

\textbf{Управление самостоятельной работой студента.}
\begin{itemize}
\item Консультации по выполнению лабораторных работ.
\end{itemize}
}


\subsubsection{Раздел 5. «Стандарты беспроводной передачи данных»}

{\parindent0pt
\setdescription{leftmargin=\parindent,labelindent=1cm}
\setitemize[1]{leftmargin=1.5cm}

\textbf{Теоретические занятия (лекции)— 5 часов.}
\begin{description}
\item[Лекция 13.] «WiFi -- IEEE 802.11». Информационная лекция. Рассматриваются следующие вопросы: \begin{itemize}
\item Стандарт IEEE 802.11. Wi-Fi\item история Wi-Fi\item применение и характеристики Wi-Fi\item стандарты a/b/g/n
\end{itemize}\item[Лекция 14.] «WiMAX -- IEEE 802.16e». Информационная лекция. Рассматриваются следующие вопросы: \begin{itemize}
\item Стандарт IEEE 802.16e WiMAX\item Особенности протокола WiMAX\item Развитие WiMAX в мире
\end{itemize}\item[Лекция 15.] «Bluetooth -- IEEE 802.15.1». Информационная лекция. Рассматриваются следующие вопросы: \begin{itemize}
\item Стандарт IEEE 802.15.1 Bluetooth\item Особенности работы Bluetooth\item Способ установки соединения\item Различия между поколениями Bluetooth
\end{itemize}\item[Лекция 16.] «NFC - ISO 13157». Информационная лекция. Рассматриваются следующие вопросы: \begin{itemize}
\item Стандарт ISO 13157\item Применение NFC
\end{itemize}\item[Лекция 17.] «LTE». Информационная лекция. Рассматриваются следующие вопросы: \begin{itemize}
\item Стандарт LTE
\end{itemize}
\end{description}




\textbf{Лабораторный практикум— 5 часов, 1 работ.}
\begin{description}
\item[Лабораторная работа 7.] «CDMA». Выполняется индивидуально в лаборатории «Компьютерный класс». Цели работы: \begin{itemize}
\item Студенты должны реализоваить визуализатор кодирования CDMA\item Студенты должны произвести эмуляцию работы с беспроводными стандартами
\end{itemize}
\end{description}

\textbf{Управление самостоятельной работой студента.}
\begin{itemize}
\item Консультации по выполнению лабораторных работ.
\end{itemize}
}



\subsubsection{Курсовые работы}
{\parindent0pt
\setdescription{leftmargin=\parindent,labelindent=0cm}
Не предусмотрены.
}

\newpage
\pagestyle{empty}
\begin{landscape}
\begin{flushright}
\textbf{Приложение 3
к рабочей программе дисциплины
«Компьютерные сети»}
\end{flushright}

Трудоемкость освоения дисциплины составляет 184 часов, из них 110 часов аудиторных занятий и 74 часов, отведенных на самостоятельную работу студента.
Рекомендации по распределению учебного времени по видам самостоятельной работы и разделам дисциплины приведены в таблице.
Формы контроля и критерии оценивания приведены в Приложениях 4 и 5 к Рабочей программе.

\begin{center}
\begin{longtable}{|p{0.3\textwidth}|p{0.6\textwidth}|c|p{0.3\textwidth}|}\hline
\multicolumn{1}{|c|}{\bfseries \pb{Вид работы}} &
\multicolumn{1}{c|}{\bfseries Содержание (перечень вопросов)} &
\bfseries Трудоемкость, час. &
\multicolumn{1}{c|}{\bfseries Рекомендации}\hline
\multicolumn{4}{|>{\columncolor[gray]{.9}}c|}{\bfseries Раздел 1. «Общие принципы построения сетей, физические и канальные протоколы»}\hline
Повторение материала лекции №1 & \begin{itemize}
\item история сред передачи данных\item передача данных по металлическим проводам\item передача данных по многожильным проводам\item передача данных по оптоволокну
\end{itemize} & 3 & См. содержимое лекции\hline
Повторение материала лекции №2 & \begin{itemize}
\item физический уровень\item канальный уровень\item сетевой уровень\item транспортный уровень\item сеансовый уровень\item представительский уровень\item прикладной уровень
\end{itemize} & 3 & См. содержимое лекции\hline
Повторение материала лекции №3 & \begin{itemize}
\item общие принципы передачи данных в сигнальной форме\item характеристики каналов связи\item физическое кодирование\item логическое кодирование
\end{itemize} & 3 & См. содержимое лекции\hline
Повторение материала лекции №4 & \begin{itemize}
\item сети Ethernet, топологии, стандарты физического и канального уровней\item сети FDDI, топология, стандарт физического и канального уровней
\end{itemize} & 3 & См. содержимое лекции\hline
Выполнение лабораторной работы №1 & \begin{itemize}
\item реализовать модули для сетевого взиамодействия, реализующие топологию шина на имеющейся сети\item реализовать модули для сетевого взиамодействия, реализующие топологию кольцо на имеющейся сети\item реализовать модули для сетевого взиамодействия, реализующие топологию звезда на имеющейся сети\item реализовать модули для сетевого взиамодействия, реализующие топологию двойное кольцо на имеющейся сети\item реализовать модули для сетевого взиамодействия, реализующие топологию решетка на имеющейся сети\item реализовать модули для сетевого взиамодействия, реализующие топологию дерево на имеющейся сети
\end{itemize} & 3 & См. задание на лабораторную работу\hline
Выполнение лабораторной работы №2 & \begin{itemize}
\item реализовать визуализатор NRZ кодирования\item реализовать визуализатор NRZI кодирования\item реализовать визуализатор манчестерского кодирования\item реализовать визуализатор RZ кодирования\item реализовать визуализатор MLT-3 кодирования\item реализовать визуализатор 4B3T кодирования
\end{itemize} & 3 & См. задание на лабораторную работу\hline
\multicolumn{1}{|r|}{Итого по разделу 1} &  & 18 & \hline
\multicolumn{4}{|>{\columncolor[gray]{.9}}c|}{\bfseries Раздел 2. «Протоколы сетевого, транспортного и прикладного уровней»}\hline
Повторение материала лекции №5 & \begin{itemize}
\item Архитектура стека TCP/IP, потоки данных через стек\item Адресация на сетевом, транспортном и прикладном уровне
\end{itemize} & 3 & См. содержимое лекции\hline
Повторение материала лекции №6 & \begin{itemize}
\item Назначение и архитектура протокола IP v 4\item Особенности протокола IP v 6\item работа протокола UDP\item работа протокола TCP
\end{itemize} & 3 & См. содержимое лекции\hline
Повторение материала лекции №7 & \begin{itemize}
\item Статическая маршрутизация в IP сетях\item Динамическая маршрутизация в IP сетях\item Трансляция адресов (NAT)\item Проксирование
\end{itemize} & 3 & См. содержимое лекции\hline
Повторение материала лекции №8 & \begin{itemize}
\item Протокол DNS\item Протоколы почтового обмена\item Протокол HTTP
\end{itemize} & 3 & См. содержимое лекции\hline
Выполнение лабораторной работы №3 & \begin{itemize}
\item Счисление адресов IP сетей\item Конфигурирование маршрутизатора
\end{itemize} & 4 & См. задание на лабораторную работу\hline
\multicolumn{1}{|r|}{Итого по разделу 2} &  & 16 & \hline
\multicolumn{4}{|>{\columncolor[gray]{.9}}c|}{\bfseries Раздел 3. «Иерархии»}\hline
Повторение материала лекции №9 & \begin{itemize}
\item предпосылки создания\item первые реализации\item устройство\item особенности протоколов
\end{itemize} & 7 & См. содержимое лекции\hline
Повторение материала лекции №10 & \begin{itemize}
\item предпосылки создания\item первые реализации\item устройство\item особенности протоколов\item сравнение PDH и SDH
\end{itemize} & 7 & См. содержимое лекции\hline
Выполнение лабораторной работы №4 & \begin{itemize}
\item разработать визуализатор метода PDH\item разработать визуализатор метода SDH
\end{itemize} & 7 & См. задание на лабораторную работу\hline
\multicolumn{1}{|r|}{Итого по разделу 3} &  & 21 & \hline
\multicolumn{4}{|>{\columncolor[gray]{.9}}c|}{\bfseries Раздел 4. «Frame relay, ATM, VLAN, PPPoE, QoS, IPv6.»}\hline
Повторение материала лекции №11 & \begin{itemize}
\item Frame relay\item ATM\item VLAN
\end{itemize} & 1 & См. содержимое лекции\hline
Повторение материала лекции №12 & \begin{itemize}
\item PPPoE\item QoS\item IPv6
\end{itemize} & 2 & См. содержимое лекции\hline
Выполнение лабораторной работы №5 & \begin{itemize}
\item Студенты должны продемонстрировать знание способов перевода адресов IPv4 в IPv6 и обратно
\end{itemize} & 2 & См. задание на лабораторную работу\hline
Выполнение лабораторной работы №6 & \begin{itemize}
\item реализовать визуализатор маршрутизатора, использующего протокол маршрутизации RIP\item реализовать визуализатор маршрутизатора, использующего протокол маршрутизации IGRP\item реализовать визуализатор маршрутизатора, использующего протокол маршрутизации BGP\item реализовать визуализатор маршрутизатора, использующего протокол маршрутизации EIGRP
\end{itemize} & 2 & См. задание на лабораторную работу\hline
\multicolumn{1}{|r|}{Итого по разделу 4} &  & 7 & \hline
\multicolumn{4}{|>{\columncolor[gray]{.9}}c|}{\bfseries Раздел 5. «Стандарты беспроводной передачи данных»}\hline
Повторение материала лекции №13 & \begin{itemize}
\item Стандарт IEEE 802.11. Wi-Fi\item история Wi-Fi\item применение и характеристики Wi-Fi\item стандарты a/b/g/n
\end{itemize} & 2 & См. содержимое лекции\hline
Повторение материала лекции №14 & \begin{itemize}
\item Стандарт IEEE 802.16e WiMAX\item Особенности протокола WiMAX\item Развитие WiMAX в мире
\end{itemize} & 2 & См. содержимое лекции\hline
Повторение материала лекции №15 & \begin{itemize}
\item Стандарт IEEE 802.15.1 Bluetooth\item Особенности работы Bluetooth\item Способ установки соединения\item Различия между поколениями Bluetooth
\end{itemize} & 2 & См. содержимое лекции\hline
Повторение материала лекции №16 & \begin{itemize}
\item Стандарт ISO 13157\item Применение NFC
\end{itemize} & 2 & См. содержимое лекции\hline
Повторение материала лекции №17 & \begin{itemize}
\item Стандарт LTE
\end{itemize} & 2 & См. содержимое лекции\hline
Выполнение лабораторной работы №7 & \begin{itemize}
\item Студенты должны реализоваить визуализатор кодирования CDMA\item Студенты должны произвести эмуляцию работы с беспроводными стандартами
\end{itemize} & 2 & См. задание на лабораторную работу\hline
\multicolumn{1}{|r|}{Итого по разделу 5} &  & 12 & \hline

\end{longtable}
\end{center}

\end{landscape}


\newpage
\pagestyle{plain}
\begin{flushright}
\textbf{Приложение 4
к рабочей программе дисциплины
«Компьютерные сети»}
\end{flushright}

\section*{ОЦЕНОЧНЫЕ СРЕДСТВА И МЕТОДИКИ ИХ ПРИМЕНЕНИЯ}

Оценивание уровня учебных достижений студента осуществляется в виде текущего контроля и промежуточной аттестации в соответствии с Положением о проведении текущего контроля успеваемости и промежуточной аттестации студентов НИУ ИТМО.

\subsection*{Фонды оценочных средств}

Фонды оценочных средств, позволяющие оценить РО по данной дисциплине, включают в себя:
\begin{itemize}
\item шаблоны отчётов по лабораторным работам, выдаются индивидуально.
\end{itemize}

\subsection*{Критерии оценивания}




\noindent\textbf{Лабораторные работы}
\textit{Допуск за защите ЛР}
Допуск к защите ЛР происходит в форме устного тестирования направленного на проверку самостоятельности выполнения ЛР. При ответе на более чем 60\% вопросов студент допускается к защите ЛР.

\noindent\textit{Защита ЛР}
Отчет по лабораторной работе представляется в электронном виде в формате, предусмотренном шаблоном отчета по лабораторной работе.
Защита отчета проходит в форме доклада студента по выполненной работе и ответов на вопросы преподавателя.
В случае если содержание и оформление отчета, а также поведение студента во время защиты соответствуют указанным требованиям, студент получает максимальное количество баллов.

Основаниями для снижения количества баллов в диапазоне от max до min являются:
\begin{itemize}
\item небрежное выполнение,
\item низкое качество графического материала.
\end{itemize}

Отчет не может быть принят и подлежит доработке в случае:
\begin{itemize}
\item отсутствия необходимых разделов,
\item отсутствия необходимого графического материала,
\item некорректной обработки результатов.
\end{itemize}




\newpage
\begin{landscape}
\begin{flushright}
\textbf{Приложение 5
к рабочей программе дисциплины
«Компьютерные сети»}
\end{flushright}


\section*{\Large Таблица планирования результатов обучения студентов 4 курса по дисциплине «Компьютерные сети» в 7 семестре}

\begin{adjustwidth}{ -0.5cm}{ -0.5cm}\begin{center}
\begin{tabular}{|c| c|c| c|c| c|c| c|c| c|c|     c|c| c|c| c|c| c|c| c|c|   c|c|}\hline
\multicolumn{1}{|c|}{\multirow{4}{*}{\pb{\bfseriesФормыконтроля}}} &
\multicolumn{10}{c|}{\pb{\bfseriesМодуль13}} &
\multicolumn{10}{c|}{\pb{\bfseriesМодуль14}} &
\multicolumn{2}{c|}{\multirow{3}{*}{\pb{\bfseriesПромежу-точнаяаттестация}}}\cline{2-21}

&
\multicolumn{8}{c|}{\pb{\bfseries Текущий контроль по точкам}} &
\multicolumn{2}{c|}{\multirow{2}{*}{\pb{\bfseries Рубежный}}} &
\multicolumn{8}{c|}{\pb{\bfseries Текущий контроль по точкам}} &
\multicolumn{2}{c|}{\multirow{2}{*}{\pb{\bfseries Рубежный}}} &
\multicolumn{2}{c|}{}\cline{2-9}\cline{12-19}

&
\multicolumn{2}{c|}{\pb{\bfseries 1}} &
\multicolumn{2}{c|}{\pb{\bfseries 2}} &
\multicolumn{2}{c|}{\pb{\bfseries 3}} &
\multicolumn{2}{c|}{\pb{\bfseries 4}} &
\multicolumn{2}{c|}{} &
\multicolumn{2}{c|}{\pb{\bfseries 1}} &
\multicolumn{2}{c|}{\pb{\bfseries 2}} &
\multicolumn{2}{c|}{\pb{\bfseries 3}} &
\multicolumn{2}{c|}{\pb{\bfseries 4}} &
\multicolumn{2}{c|}{} &
\multicolumn{2}{c|}{} \cline{2-23}

&
\pb{\tiny min} &
\pb{\tiny max} &
\pb{\tiny min} &
\pb{\tiny max} &
\pb{\tiny min} &
\pb{\tiny max} &
\pb{\tiny min} &
\pb{\tiny max} &
\pb{\tiny min} &
\pb{\tiny max} &

\pb{\tiny min} &
\pb{\tiny max} &
\pb{\tiny min} &
\pb{\tiny max} &
\pb{\tiny min} &
\pb{\tiny max} &
\pb{\tiny min} &
\pb{\tiny max} &
\pb{\tiny min} &
\pb{\tiny max} &

\pb{\tiny min} &
\pb{\tiny max} \hline

\pb{Лабораторная работа} &  &  & 7 & 13 &  &  & 9 & 14 &  &  &  &  &  &  &  &  & 14 & 23 &  &  &  & \hline
\pb{Рубежное тестирование} &  &  &  &  &  &  &  &  & 6 & 10 &  &  &  &  &  &  &  &  & 6 & 10 &  & \hline
\pb{Личностные качества} &  &  &  &  &  &  & 3 & 5 &  &  &  &  &  &  &  &  & 3 & 5 &  &  &  & \hline
\pb{Экзамен} &  &  &  &  &  &  &  &  &  &  &  &  &  &  &  &  &  &  &  &  & 12 & 20\hline
\pb{Балловая стоимостьодной точки} & 0 & 0 & 7 & 13 & 0 & 0 & 12 & 19 & 6 & 10 & 0 & 0 & 0 & 0 & 0 & 0 & 17 & 28 & 6 & 10 & 12 & 20\hline
\pb{Накопление баллов} & 0 & 0 & 7 & 13 & 7 & 13 & 19 & 32 & 25 & 42 & 0 & 0 & 0 & 0 & 0 & 0 & 17 & 28 & 23 & 38 &  & \hline

\multicolumn{21}{|r|}{\pb{\bfseries Итого:}} &60 &  100\hline
\end{tabular}
\end{center}\end{adjustwidth}


\section*{\Large Таблица планирования результатов обучения студентов 4 курса по дисциплине «Компьютерные сети» в 8 семестре}

\begin{adjustwidth}{ -0.5cm}{ -0.5cm}\begin{center}
\begin{tabular}{|c| c|c| c|c| c|c| c|c| c|c|     c|c| c|c| c|c| c|c| c|c|   c|c|}\hline
\multicolumn{1}{|c|}{\multirow{4}{*}{\pb{\bfseriesФормыконтроля}}} &
\multicolumn{10}{c|}{\pb{\bfseriesМодуль15}} &
\multicolumn{10}{c|}{\pb{\bfseriesМодуль16}} &
\multicolumn{2}{c|}{\multirow{3}{*}{\pb{\bfseriesПромежу-точнаяаттестация}}}\cline{2-21}

&
\multicolumn{8}{c|}{\pb{\bfseries Текущий контроль по точкам}} &
\multicolumn{2}{c|}{\multirow{2}{*}{\pb{\bfseries Рубежный}}} &
\multicolumn{8}{c|}{\pb{\bfseries Текущий контроль по точкам}} &
\multicolumn{2}{c|}{\multirow{2}{*}{\pb{\bfseries Рубежный}}} &
\multicolumn{2}{c|}{}\cline{2-9}\cline{12-19}

&
\multicolumn{2}{c|}{\pb{\bfseries 1}} &
\multicolumn{2}{c|}{\pb{\bfseries 2}} &
\multicolumn{2}{c|}{\pb{\bfseries 3}} &
\multicolumn{2}{c|}{\pb{\bfseries 4}} &
\multicolumn{2}{c|}{} &
\multicolumn{2}{c|}{\pb{\bfseries 1}} &
\multicolumn{2}{c|}{\pb{\bfseries 2}} &
\multicolumn{2}{c|}{\pb{\bfseries 3}} &
\multicolumn{2}{c|}{\pb{\bfseries 4}} &
\multicolumn{2}{c|}{} &
\multicolumn{2}{c|}{} \cline{2-23}

&
\pb{\tiny min} &
\pb{\tiny max} &
\pb{\tiny min} &
\pb{\tiny max} &
\pb{\tiny min} &
\pb{\tiny max} &
\pb{\tiny min} &
\pb{\tiny max} &
\pb{\tiny min} &
\pb{\tiny max} &

\pb{\tiny min} &
\pb{\tiny max} &
\pb{\tiny min} &
\pb{\tiny max} &
\pb{\tiny min} &
\pb{\tiny max} &
\pb{\tiny min} &
\pb{\tiny max} &
\pb{\tiny min} &
\pb{\tiny max} &

\pb{\tiny min} &
\pb{\tiny max} \hline

\pb{Лабораторная работа} &  &  &  &  &  &  & 16 & 27 &  &  &  &  & 4 & 7 & 5 & 8 & 5 & 8 &  &  &  & \hline
\pb{Рубежное тестирование} &  &  &  &  &  &  &  &  & 6 & 10 &  &  &  &  &  &  &  &  & 6 & 10 &  & \hline
\pb{Личностные качества} &  &  &  &  &  &  & 3 & 5 &  &  &  &  &  &  &  &  & 3 & 5 &  &  &  & \hline
\pb{Экзамен} &  &  &  &  &  &  &  &  &  &  &  &  &  &  &  &  &  &  &  &  & 12 & 20\hline
\pb{Балловая стоимостьодной точки} & 0 & 0 & 0 & 0 & 0 & 0 & 19 & 32 & 6 & 10 & 0 & 0 & 4 & 7 & 5 & 8 & 8 & 13 & 6 & 10 & 12 & 20\hline
\pb{Накопление баллов} & 0 & 0 & 0 & 0 & 0 & 0 & 19 & 32 & 25 & 42 & 0 & 0 & 4 & 7 & 9 & 15 & 17 & 28 & 23 & 38 &  & \hline

\multicolumn{21}{|r|}{\pb{\bfseries Итого:}} &60 &  100\hline
\end{tabular}
\end{center}\end{adjustwidth}



\end{landscape}

